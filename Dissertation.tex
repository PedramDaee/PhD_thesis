%% Select the dissertation mode on
%
% See the documentation for more information about the available class options 
% ('math', 'vertlayout', 'pdfa', ...)
% If you give option 'draft' or 'draft*', the draft mode is turned on
% NOTE if you want to generate abstracts for the publication platform, use
% the option 'abstracts'!
% The pdfa option is experimental, but give it a try -- your doc will be better archivable
%
\documentclass[dissertation,math,vertlayout,pdfa,colorlinks]{aaltoseries}

% Kludge to make sure we have utf8 input (check that this file is utf8!)
\makeatletter
\@ifpackageloaded{inputenc}{%
  \inputencoding{utf8}}{%
  \usepackage[utf8]{inputenc}}
\makeatother

% for live links. Takes the above option 'colorlinks' (use 'hidelinks' if you want them black for print).
\usepackage{hyperref} 

% Lipsum package generates quasi latin filler text
\usepackage{lipsum}
% Set the document languages used
\usepackage[english]{babel}
% more math symbols and environments if needed
\usepackage{amsmath,amssymb,amsthm} 
% after amsmath to restore bad page breaks in the middle of equations... only for those in the know
\interdisplaylinepenalty=2500 
% Adjust math line spacing
\renewcommand*{\arraystretch}{1.2} % for array/matrix environments
\setlength{\jot}{8pt} % for split environment

\usepackage{listings} % neat printing of source code
\usepackage[indentfirst=false,vskip=3mm]{quoting} % flexible quotes and quotations
% Enable the following to suppress page headers and numbers on 
% content-less left (even-numbered) pages. Fixes a bug in aaltoseries
\usepackage{emptypage}
\usepackage[nobiblatex]{xurl}


%Mathematical notations used 
\usepackage{amsmath}
\usepackage{amsfonts}
\usepackage{mathtools}
\usepackage{bm}

\newcommand{\bw}{\bm{w}}
\newcommand{\bX}{\bm{X}}
\newcommand{\by}{\bm{y}}
\newcommand{\bD}{\mathcal{D}}
\DeclareMathOperator{\eye}{\textbf{I}}

\newcommand{\KL}{\KLt\infdivx}
\DeclareMathOperator{\betapdf}{Beta}
\DeclareMathOperator{\bernoullipdf}{Bernoulli}
\DeclareMathOperator{\normalpdf}{N}
\DeclareMathOperator{\gammapdf}{Gamma}
\DeclareMathOperator{\new}{new}
\DeclareMathOperator{\old}{old}
\DeclareMathOperator{\E}{\mathbb{E}}
\DeclareMathOperator{\var}{var}
\newcommand{\tp}{^{\top}}
%%
%% There's a HUGE number of LaTeX packages. Whatever it is you need, search for a package first before rolling your own!
%%

% This is the way you may input and separately develop your individual chapters. Write, e.g., 
%
%	%\input{Ch1}
%	%\input{Ch2}
%	\input{Ch3}
%	%\input{Ch4}
%	%\input{Ch5}
%
% ...when editing only the third chapter. Compilation with pdflatex ('pdflatex dissertation') will then only output
% a thin dissertation containing only the third chapter, but properly formatted.
%
% You may leave of the .tex extension here...
%
%    Some definitions useful in producing this sort of documentation:
\chardef\bslash=`\\ % p. 424, TeXbook
%    Normalized (nonbold, nonitalic) tt font, to avoid font
%    substitution warning messages if tt is used inside section
%    headings and other places where odd font combinations might
%    result.
\newcommand{\ntt}{\normalfont\ttfamily}
%    command name
\newcommand{\cn}[1]{{\protect\ntt\bslash#1}}
%    LaTeX package name
\newcommand{\pkg}[1]{{\protect\ntt#1}}
%    File name
\newcommand{\fn}[1]{{\protect\ntt#1}}
%    environment name
\newcommand{\env}[1]{{\protect\ntt#1}}
\hfuzz1pc % Don't bother to report overfull boxes if overage is < 1pc

%       Theorem environments

%% \theoremstyle{plain} %% This is the default
\newtheorem{thm}{Theorem}[section]
\newtheorem{cor}[thm]{Corollary}
\newtheorem{lem}[thm]{Lemma}
\newtheorem{prop}[thm]{Proposition}
\newtheorem{ax}{Axiom}

\theoremstyle{definition}
\newtheorem{defn}{Definition}[section]

\theoremstyle{remark}
\newtheorem{rem}{Remark}[section]
\newtheorem*{notation}{Notation}

%\numberwithin{equation}{section}

\newcommand{\thmref}[1]{Theorem~\ref{#1}}
\newcommand{\secref}[1]{\S\ref{#1}}
\newcommand{\lemref}[1]{Lemma~\ref{#1}}

\newcommand{\bysame}{\mbox{\rule{3em}{.4pt}}\,}

%       Math definitions

\newcommand{\A}{\mathcal{A}}
\newcommand{\BB}{\mathcal{B}}
\newcommand{\st}{\sigma}
\newcommand{\XcY}{{(X,Y)}}
\newcommand{\SX}{{S_X}}
\newcommand{\SY}{{S_Y}}
\newcommand{\SXY}{{S_{X,Y}}}
\newcommand{\SXgYy}{{S_{X|Y}(y)}}
\newcommand{\Cw}[1]{{\hat C_#1(X|Y)}}
\newcommand{\GG}{{G(X|Y)}}
\newcommand{\PY}{{P_{\mathcal{Y}}}}
\newcommand{\X}{\mathcal{X}}
\newcommand{\wt}{\widetilde}
\newcommand{\wh}{\widehat}

\DeclareMathOperator{\per}{per}
\DeclareMathOperator{\cov}{cov}
\DeclareMathOperator{\non}{non}
\DeclareMathOperator{\cf}{cf}
\DeclareMathOperator{\add}{add}
\DeclareMathOperator{\Cham}{Cham}
\DeclareMathOperator{\IM}{Im}
\DeclareMathOperator{\esssup}{ess\,sup}
\DeclareMathOperator{\meas}{meas}
\DeclareMathOperator{\seg}{seg}

%    \interval is used to provide better spacing after a [ that
%    is used as a closing delimiter.
\newcommand{\interval}[1]{\mathinner{#1}}

%    Notation for an expression evaluated at a particular condition. The
%    optional argument can be used to override automatic sizing of the
%    right vert bar, e.g. \eval[\biggr]{...}_{...}
\newcommand{\eval}[2][\right]{\relax
  \ifx#1\right\relax \left.\fi#2#1\rvert}

%    Enclose the argument in vert-bar delimiters:
\newcommand{\envert}[1]{\left\lvert#1\right\rvert}
\let\abs=\envert

%    Enclose the argument in double-vert-bar delimiters:
\newcommand{\enVert}[1]{\left\lVert#1\right\rVert}
\let\norm=\enVert



% The author of the dissertation
\author{Pedram Daee}
% The title of the thesis
\title{Interactive user modelling for human-in-the-loop machine learning} % This is the title of our line of work as a group.

%Comments from meeting with Sami on 15.07.2019:
% "user modelling" may not be the best term as people may perceive it as a statitstic UM. But "probabilistic user modelling" in a good context may be OK.
%  We want to have User + probabilistic in the title.
% Alternatives for user could be human-in-the-loop or user interaction or HCI or modellign of the user or interactive user modelling
% an alternative title could be to directly mention what I did: knowledge elicitation or multi source feedback. Or use these as subtitles?


%Keywords: Interactive intent modelling, User modelling, Interactive knowledge elicitation, Active elicitation of knowledge, Sequential probabilistic inference, high dimensional prediction.
%0. Probabilistic user modelling methods for improving human-in-the-loop machine learning
%1.	Probabilistic User Modelling in Interactive Machine Learning for Prediction
%2.	Machine learning methods for user modelling for improved prediction.
%3.	Interactive user/expert modelling for improved prediction 
%4.	Machine Learning Methods for Improved Prediction through Probabilistic User Modelling
%5.	Improved Prediction through Interactive Probabilistic User Modelling
%6.	Probabilistic user modelling methods for improving human-in-the-loop machine learning.

%Other comments: I do not like knowledge elicitation since it seems like a term that may become obsolete through time

\begin{document}

%% The abstract of the dissertation in English
% Use this command!
%\draftabstract{\lipsum[1-2]}%
%\draftabstract[english]{\hspace{-2pt} My abstract in English}
% Let's add another one in Finnish
%\draftabstract[finnish]{\hspace{-2pt} En puhu suomea1

%}%


% And yet another one in Swedish
%\draftabstract[swedish]{\lipsum[7-9]}
%%---------------------

%% The abstract of the dissertation in English
% Use this command!
%\begin{abstract}\lipsum[1-3]\end{abstract}%
% Let's add another one in Finnish

\begin{abstract}[english]
	Ongoing title: \\
	\textbf{Interactive user modelling for human-in-the-loop machine learning}
\end{abstract}

%1. What is the bigger picture?
%2. Dissertation purpose
%3. Research method
%4. Key results
%5. Practical Implications

%Typically 3 paragraphs: 1. the introduction of the challenge problem, 2. thesis, 3. contribution

 
\begin{abstract}%[finnish]
En puhu suomea.
\end{abstract}


% And yet another one in Swedish
%\begin{abstract}[swedish]\lipsum[7-9]\end{abstract}


%% Preface
% If you write this somewhere else than in Helsinki, use the optional location.
\begin{preface}[Espoo]
To fill.
%Maybe consider whom you should acknowledge. 
% at the very least, all the co-authors, group members who have worked in this or have acknowledge you in their theses, and the funding agencies (reknow, MindSee, WAI?)

\end{preface}

%% Table of contents of the dissertation
\clearpage
\tableofcontents

% To be defined before generating list of publications. Leave off if no acknowledgement
%\languagecheck{the Institute of Language Checks}

%% This is for article dissertations. Remove if you write a monograph dissertation.
% The actual publications are entered manually one by one as shown further down:
% use \addpublication, \addcontribution, \adderrata, and addpublicationpdf.
% The last adds the actual article, the other three enter related information
% that will be collected in lists -- like this one.
\listofpublications

%% Add lists of figures and tables as you usually do (\listoffigures, \listoftables)

%%%%%%%%%%%%%%%%%%%%%%% TODO: do I want to have the followings? Fig, table, Abb, sym? others did not have it.
%\listoffigures
%\listoftables

%\abbreviations
%\begin{description}
%\item[PDF] Probability Density Function
%\end{description}

%\symbols
%\begin{description}
%\item[$p(\theta|y)$] Conditional distribution
%\end{description}


%% The main matter, one can obviously use \input or \include
\chapter{Introduction}
%Comments from meeting with Sami on 15.07.2019:
% I need to tie to non-ML related works somewhere in the paper. This can include HCI related works or knowledge elicitation or multi agent? I need to either list them here and give a short descriotion or put them inside the next two chapters. In any case, I need to refer back to these in the Discussion section and mention how we improved the fields (and how we benefited from these).
% I need to also explain why probabilistic modelling is good 




%We can keep this empty and start from next. Alternatively we can write some stuff here and have only one section as "Contributions and organization of the thesis". See which one works better for you


Whether it is an everyday user searching for an application in her mobile phone or a doctor working with a cancer diagnostic system, humans and machines are increasingly interacting with each other. The goal of the thesis is to improve this interaction by incorporating a model of the human user in the system they are interacting with. In particular, the thesis considers the family of problems where the human and machine interact to solve a prediction problem. Such problems can include personalized search activity or medical prediction about the response of a cancer drug. An important common factor in both these scenarios is that the number of labeled data (training data) that the machine can use to make predictions, is usually very few compared to the dimension of search space. This results in ill-posed statistical learning since there are limits in how low in sample size statistical methods can go \cite{Donoho2009observed}. 


%\paragraph{Objective.}
%The objective of the thesis is to tackle the limited data problem by directly integrating a model of the human user into the modelling loop. The user can then provide more information about the problem (for example by giving feedback in response to the machine's query) that can help to improve the prediction performance. For example, in precision medicine obtaining additional data can be extremely costly or even impossible but the human knowledge, e.g., practitioners feedback, can be available and used to improve the prediction. My thesis proposes new probabilistic machine learning models to learn the human intent in interaction, new user models to learn about the knowledge of the human and accounting for common human biases, and new prediction models to employ and extract human knowledge to improve a task.


%\paragraph{Methodology.}
%USED FOR CHAPTER 2: I design probabilistic machine learning models to tie the data model (prediction model from the training data) to the user model (the model of the human) in a unified way. After doing so, we use Bayesian inference to find the posterior, i.e, distribution of the unknown parameters related to data and user, giving the training data and user interaction. The posterior inference, in most cases, does not have an analytical solution. We use different approximation methods, such as expectation propagation and variational inference, to handle the computation. 

%A core characteristic of the formulation is that the model adapts to the feedback obtained from the user and it sequentially integrates every piece of information before deciding on the next query for the user. This is important as humans can only answer limited number of queries (out of several thousands). The query selection is naturally formulated as experimental design problem, aiming at maximizing the most information gained, or multi-armed bandit problem, aiming at finding the most important piece of information, depending on the targeted task. For evaluation and testing, we conduct carefully designed user studies with real human.  



\section{Motivation}

%Thinking back, I think my main motivation should be on how to tackled limited interaction problem. In iui2016 we did it by adding new likelihoods, in JASIS by implicit interaction, in ML and IUI 18 by assuming strong priors.

%A) Small number of data ->add users

%B) user is the source of data

%Limited interaction: 1) need to be smart about what question to ask from the user 2) Should be able to consider all potential sources of feedback on different components of the interaction framework.


 
\section{Research questions and contributions}


This thesis investigates methods to tackle the limited user interaction challenge in interactive machine learning for prediction. The thesis focuses on scenarios where there is few labeled data available compared to the dimension of the problem, or when a human user is provider of the labeled data. The core idea of the thesis is to jointly model the human user with the data as part of a unified probabilistic model and use the model to improve the interaction.\\

\noindent \textbf{RQ1 --} \textit{Can we exploit new sources of interaction as additional learning signals from human user to improve interactive intent modelling?}

Publications I and V contribute to this research question by proposing models to incorporate new types of user feedback to amend the limited feedback in exploratory information seeking tasks. The tasks considered are document search scenarios where a user needs to sequentially provide relevance feedback to suggested keywords in order to find the targeted document. This is modelled as a multi-armed bandit problem with the goal of finding the most relevant document with minimum interaction. In particular, Publication I couples user relevance feedback on both documents and keywords by assuming a shared underlying latent model connected through a probabilistic model of the relationship between keywords and documents. Thompson sampling on the posterior of the latent intent was then used to recommend new documents and keywords in each iteration. Publication V investigates the use of implicit relevance feedback from neurophysiology signals for effortless information seeking. The work contributes by demonstrating how to integrate this inherently noisy and implicit feedback source with scarce explicit interaction. A model for controlling the accuracy of the feedback given its nature (implicit or explicit) was introduced. Similar to Publication I, Thompson sampling was used to control the exploration and exploitation balance of the recommendations. Both publications were evaluated by user studies in realistic information seeking tasks. 


\noindent \textbf{RQ2 --} \textit{Can expert knowledge about high dimensional data models be elicited to improve the prediction performance?}

Publications II and III contribute to this research question. Publication II proposes a framework for user knowledge elicitation as a probabilistic inference process, where the user knowledge is sequentially queried to improve predictions. In particular, sparse linear regression is considered as the data model with access to only few high-dimensional training data. It is assumed that there are experts who have knowledge about the relevance of the covariates, or of values of the regression coefficients and can provide this information to the data model if queried. The work contributes by an algorithm and computational approximation for fast and efficient interaction, which sequentially identifies the most informative queries to ask from the user. Publication III, builds on Publication II by adding user knowledge about direction of relevance of covariates and applying the method in important applications of precision medicine with the goal of predicting the effects of different treatments using high-dimensional genomic measurements. Both publications were evaluated by extensive simulations and user studies. 
Source codes for methods presented in Publications II and III, and user study data from Publication II are available at \\ \texttt{https://github.com/HIIT/knowledge-elicitation-for-linear-regression} and \\ \texttt{https://github.com/AaltoPML/knowledge-elicitation-for-precision-medicine}.




\noindent \textbf{RQ3 --} \textit{Is it enough to incorporate human knowledge directly in the data model as explained in RQ2, or could it be beneficial to account for rational knowledge updates that humans may undergo during the interaction?}

Publication IV contributes to this research question by modelling the knowledge provider, here the human user, as a rational agent that updates its knowledge about the underlying prediction task during the interaction. In particular, certain aspects of training data may be revealed to the user during knowledge elicitation. \textcolor{red}{The design of the system is then critical, since the elicited user knowledge cannot be assumed to be independent from the data model knowledge coming from the training data. If not accounted properly, knowledge elicitation can lead to double use of data and overfitting, if the user reinforces noisy patterns in the data. We propose a user modelling methodology, by assuming simple rational behaviour, to correct the problem and evaluate the method in a user study.} 
Source code and user study data are available at \texttt{https://github.com/HIIT/human-overfitting-in-IML}.

\section{Organization of the thesis}

The organization of the thesis is as follows. Chapter 2 provides an overview of probabilistic modelling and introduces our approach of modelling the user and \textcolor{red}{data} as a joint probabilistic model. Chapter three investigates the purpose of interactions and reviews different utility functions designed to minimize the effort of the user. The fourth chapter summarizes Publications I-IV. Chapter five concludes the thesis and provides discussions for future works.


%Comments from meeting with Sami on 15.07.2019:
% "data" usage is a bit awkward here as we are also modelling user interaction as new data. Maybe try to fine a better term?

\chapter{Probabilistic modelling of data and user} \label{prob_model_data_user}
% Sami's comments 15.07.2019: I need to also explain why probabilistic modelling is good and our approach for modelling the world. 

%I need to tie the chapters to RQs like what Antti did if I want to go with RQ type of thesis. "This chapter gives an overview of the line of research related to RQ1"

This chapter provides a brief introduction to probabilistic modelling as the main statistical framework that is used through the thesis. After some preliminaries,  Section \ref{prob_model_data} briefly reviews the type of linear models that are used in Publication I-IV for prediction. %  investigates linear models that are commonly used  for prediction.
Section \ref{prob_model_user} introduces different types of user interaction with the linear model and explains how user knowledge about the model can be incorporated as observational feedback. Finally, the key components for modelling observational data and user feedback as a joint probabilistic model is introduced in Section \ref{key_comp}.

\section{Preliminaries}
The core idea of probabilistic modelling is to describe all the unobserved parameters and observed data as random variables from probability distributions. The unobserved parameters include the unknown quantity of interest or other parameters that affect the data or the quantity of interest. Bayesian inference provides a powerful framework to fit the described probabilistic model to observational data \cite{Gelman2013}. A core feature of Bayesian inference is that it provides probability distributions as the solution, compared to deterministic methods which provide a single outcome. This uncertainty quantification is of particularly high interest in cases where few observational data are available or when the data acquisition scheme is controlled by the model. Both of these constrains are prominently present in the tasks investigated by this thesis.

We follow the notation of \cite{Gelman2013} and use $p(.)$ to denote a probability distribution and $p(.|.)$ a conditional distribution. Consider the case where there are a set of observations $\bD= \{y_1,\ldots,y_n\}$ generated from a probabilistic model with an unobserved parameter of interest $\theta$. The observational model for an observation $y$ describes the conditional density of $y$ given the parameter $\theta$ and is denoted as $y \sim p(y \mid \theta)$. It is usually assumed that the observations are conditionally independent given $\theta$, enabling us to write the model for all observations as $p(\bD \mid \theta) =  \prod_{i=1}^{n} p(y_i \mid \theta)$. The observational model is called likelihood function if perceived as a function of $\theta$ with fixed observation $y$. One of the core questions in statistical inference is to estimate the parameter of interest $\theta$ based on observations $\bD$. % and the likelihood assumption. 
Bayesian inference answers this question by computing the conditional distribution of $\theta$ given $\bD$ following the Bayes rule


\begin{equation}\label{Eq:Bayes}
p(\theta \mid \bD) = \frac{p(\bD \mid \theta)p(\theta)}{p(\bD)}.
\end{equation}  

Where $p(\theta)$ represents the prior belief about $\theta$ and $p(\bD)$ is called the marginal likelihood and acts as a normalization factor as it does not depend on $\theta$. The marginal likelihood can be computed using the marginalization rule, i.e., $p(\bD)= \int_{\theta} p(\bD, \theta)d\theta= \int_{\theta} p(\bD \mid \theta)p(\theta) d\theta$ \footnote{The integral turns to summation for discrete $\theta$.}. $p(\theta \mid \bD)$ is called the posterior and it expresses the uncertainty surrounding the true value of $\theta$, after updating the prior assumptions about $\theta$ (i.e., the prior distribution) with the knowledge coming from the observations through the likelihood.  

In many cases, we may be more interested to make a prediction about an unknown observable data point $\hat{y}$ rather than the parameter $\theta$. Bayesian inference  allows us to compute the conditional distribution of this unknown observable data given the observed data point as 

\begin{equation}
p(\hat{y} \mid\bD ) = \int_{\theta} p(\hat{y}  \mid \theta)p(\theta \mid \bD)d\theta. 
\end{equation}

$p(\hat{y} \mid \bD)$ is known as the posterior predictive distribution and represents the uncertainty about a potential new observation. An example usage of this distribution could be when we want to predict the value of a test data. Since test data is not observed, we can use posterior predictive distribution as our best guess. However, in many applications, only a value (and not a distribution) is required as the prediction. This can be handled by using some statistics of the distribution (for example mean or median) as the prediction. Still, the quantified uncertainty in the distribution can be useful as it provides knowledge about how certain we are about our estimate. Particularly, this uncertainty can help us to design more efficient interaction with a user, as will be discussed in Section \ref{interaction}.

	


\section{Modelling data for prediction} \label{prob_model_data}
%Alternative title: Prediction of high-diensional data?

Prediction is one of the core problems in statistical analysis and supervised machine learning. Given a set of $n$ input and output pairs, called training data, denoted by $\bD= \{(\bm{x}_i,y_i)\}_{i=1}^{n}$, the goal is to find a mapping from inputs to outputs. Here, $\bm{x}_i = [x_i^1,\ldots,x_i^d]\tp \in \mathbb{R}^d$ is a d-dimensional column vector representing the values of $i^{th}$ data point \footnote{We use subscripts to index an specific item (e.g, one particular observation out of several) and superscripts to refer to dimensions of the variable.}. The dimensions are commonly called feature, covariates, or attribute. $y_i$ is the corresponding response (or target) variable which can be anything depending on the underlying problem. In this thesis, we consider the regression tasks, meaning that we model response variables by real values, i.e., $y_i \in \mathbb{R}$. The problem is called classification in supervised learning if the response variable is restricted to categorical values (sometimes called classes). 
  
A well-studied and widely practical type of regression, known as linear regression, assumes that the relationship between all inputs and their corresponding response variables is linear. This relationship can be written as 

\begin{equation*}
y_i= \sum_{j=1}^{d}x_{i}^{j} w^{j}+\epsilon_i=\bm{x}_i\tp\bm{w}+\epsilon_i, \qquad i=1,\ldots,n, 
\end{equation*}

\noindent where $\bm{w} \in \mathbb{R}^d$ is the regression coefficients or the model's weights and $\epsilon_i$ is the residual error between linear prediction $\bm{x}_i\tp\bm{w}$ and the response value $y_i$. Given the labeled data $\bD$, a commonly used error function for this problem is the sum of squared of residual errors $\sum_{i=1}^n(\bm{x}_i\tp\bm{w}- y_i)^2$. By stacking the inputs in $\bX = [x_1,\ldots,x_n]\tp \in \mathbb{R}^{n\times d}$ and outputs in $\by = [y_1,\ldots,y_n]\tp \in \mathbb{R}^n$ the error can be written as $(\bX\bm{w}- \by)\tp(\bX\bm{w}- \by)$. The frequentist solution directly finds the point estimate for $\bw$ that minimizes this error. It is straightforward to show that $\hat{\bw} = (\bX \tp \bX)^{-1}\bX\tp \by$  would be the point estimate solution given that $(\bX \tp \bX)^{-1}$ is invertible.  

However, as mentioned, we are interested in directly quantifying the uncertainty of the solution. The probabilistic way to model this problem is to explicitly describe the model assumptions (likelihood and priors) as probability distributions. In linear regression, the residual errors $\epsilon_i$ and the weights $\bm{w}$ are modelled as random variables and the inputs $\bm{x}_i$ as vector of values that are given. A customary assumption is to model the residual errors as independent zero-mean Gaussian random variables $\epsilon_i \sim \normalpdf(0,\sigma^2)$, where $\sigma^2$ is variance of the Gaussian distribution which indicates the model tolerance about residual errors. It is common to also model $\sigma^2$ as another random variable with its own distribution assumption, however, we can consider it to be a fixed hyperparameter for simplicity for now. The quantity of interest in the linear model is the regression coefficients. To complete the Bayesian inference loop, we need to consider a prior distribution on $\bm{w}$. There are many ways to do this depending on the underlying task. One simple prior could be to assume that coefficients are independent and each come from a zero-mean Gaussian distribution, encouraging the weights to be close to zero. This simple Bayesian linear regression can be described by

\begin{align*} 
\by \sim \normalpdf(\bX\bm{w},\sigma^2 \eye)\\
\bm{w} \sim \normalpdf(0,\tau^2 \eye), 
\end{align*}

\noindent where $\tau^2$ is the variance of the weights and $\eye$ is the identity matrix. For simplicity we assume for now that $\tau^2$ is also a fixed hyperparameter. Therefore, The only unobserved parameters of this model is $\bm{w}$. The Bayesian rule (Equation \ref{Eq:Bayes}) allows us to derive the posterior for $\bm{w}$ as

\begin{equation}
p(\bm{w} \mid \bD) = \frac{p(\bD \mid \bm{w})p(\bm{w})}{p(\bD)}.
\end{equation}  
%p(\bm{w} \mid \bX , \by) = \frac{\normalpdf(\by \mid \bX \bm{w},\sigma^2 \eye) \normalpdf(\bw \mid 0,\tau^2 \eye))}{p(\bX , \by)}.

Generally, the posterior in many problems cannot be analytically derived (we will discuss this issue more in XXX). For this simple model, however, an analytical solution is available. We will go through the steps as a hand on with posterior inference. Writing down the distribution assumptions and dropping $p(\bD)$ (as it is a constant factor) gives

\begin{equation}
p(\bm{w} \mid \bD) \propto \prod_{i=1}^{n} \normalpdf(y_i|\bm{x}_i\tp\bm{w},\sigma^2) \normalpdf(\bm{w}|0,\tau^2 \eye).
\end{equation}    
 
\noindent Note that we use for example $\bm{w} \sim \normalpdf(0,\tau^2 \eye)$ to denote the random variable $\bm{w}$ and $\normalpdf(\bm{w}|0,\tau^2 \eye)$ to refer to the density function. %todo: THIS IS OUT OF PLACE

HERE GOES THE REST OF STEPS



\begin{equation}
\by \sim \normalpdf(\bX \bw, \sigma^2 \eye),
\end{equation}
ALSO EXPLAIN BRIEFLY THE NON BAYESIAN TREATMENT AND EXPLAIN WHY WE GO WITH BAYES/
%The role of model (prior) assumptions is crucial then since in all scenarios considered in this thesis there are only few observed data available which is usually smaller than the dimension of the problem (e.g., feature dimension)...

%Defining appropriate likelihood and defining appropriate priors
%GO through linear models here and maybe briefly mention and cite generalized linear models. 
	


\subsection{Bayesian Linear regression}
% Here we want to start with linear regression (ridge) and then talk about sparsifying priors and maybe automatic relevance models? (relevant thesis: Tomi and Juho)
\subsection{Sparse priors}

\section{Modelling the user}\label{prob_model_user}

%Our general approach is to consider new likelihood function for the user feedback and put appropriate priors on top of that. Explain the feedback likelihoods?




\section{Key components of the joint model}\label{key_comp}



%\item Idea: Have a picture per section where there is a cloud on top of user and system head and in it there is the plate diagram of the particular model. Then talk about the interaction. 
%In the first introduction section have the same figure but just write down (or explain) user probabilistic model and data probabilistic model (priors) and arrows for interactions? and then refer to later chapters for the type of interaction

%\item Then explain the two possible likelihoods and show a diagram where there is a section for data modeling and there is a section for user modeling that are connected through a shared latent parameter and then each have their own models. Then explain that we will go through details of the data and user model in the next two subsections. My contribution comes from adding data likelihoods and user priors and having interaction.

%I design probabilistic machine learning models to tie the data model (prediction model from the training data) to the user model (the model of the human) in a unified way. After doing so, we use Bayesian inference to find the posterior, i.e, distribution of the unknown parameters related to data and user, giving the training data and user interaction. The posterior inference, in most cases, does not have an analytical solution. We use different approximation methods, such as expectation propagation and variational inference, to handle the computation.


\textcolor{red}{Let $y$ and $x$ denote the outputs (target variables) and inputs (covariates), and $\theta$ and $\phi_y$ the model parameters. Let $f$ encode input (\textit{feedback}) from the user, presumably a domain expert, and $\phi_f$ be model parameters related to the user input. We identify the following key components:
	%
	\begin{enumerate}%[leftmargin=*,noitemsep,topsep=0pt]
		\item An observation model $p(y\mid x,\theta,\phi_y)$ for $y$. %data model
		\item A feedback model $p(f\mid\theta, \phi_f)$ for the expert's knowledge. % user model
		\item A prior model $p(\theta, \phi_y, \phi_f)$ completing the hierarchical model description.
		\item A query algorithm and user interface that facilitate gathering $f$ iteratively from the expert.
		\item Update process of the model after user interaction.
	\end{enumerate}
	%
	The observation model can be any appropriate probability model. It is assumed that there is some parameter $\theta$, possibly high-dimensional, that the expert has knowledge about. The expert's knowledge is encoded as (possibly partial) feedback $f$ that is transformed into information about $\theta$ via the feedback model. Of course, there could be a more complex hierarchy tying the observation and feedback models, and the feedback model can also be used to model more user-centric issues, such as the quality of or uncertainty in the knowledge or user's interests.}

\chapter{User interaction with the probabilistic model}\label{interaction}

%Here have the same figure, explain how user feedback is gathered. explain that we consider two sets of works. One where there is external data other than user feedback and one where the user feedback IS the data. Then explain our interaction goals and acquisition functions and experimental design.

\section{Active learning and experimental design}

\section{Multi-armed bandits and Bayesian optimization}

\chapter{Summary of the Contributions}
%Applications and results OR Summary of the Publications
%Go through the papers and tasks. 
This chapter briefly summarizes the contributions of Publications I-V with emphasize on answering the research questions of the thesis. 

\section{Interactive intent modelling from multiple feedback domains (Publications I and V)}

\section{Expert knowledge elicitation for high-dimensional prediction (Publications II and III)}

\section{User modelling for avoiding overfitting in knowledge elicitation (Publication IV)}


\chapter{Discussion}


%Motivate future works such as next level of user modelling. Similar to overfitting work, but assuming a more active version of the user -> ATOM





%--------------------------------END--------------------------%


 % Refer to the Journal paper 1 of this example document
%\citepub{j1} \& \cpub{j1} \& \cp{j1} \& \pageref{j1} \& \ref{j1}
% Refer to the Conference paper of this example document
%\citepub[p.~2]{c1} \& \cpub[Sec.~ 1]{c1} \&  \cp[pp.~1--2]{c1} \& \pageref{c1} \& \ref{c1} 


\renewcommand{\bibname}{References}
\bibliographystyle{plain} % Change as required
\LARGE\bibliography{references}  % remember to edit the file name




%% The following commands are for article dissertations, remove them if you write a monograph dissertation.

% Errata list, if you have errors in the publications.
%\errata

%% The first publication (journal article)
% Set the publication information.
% This command musts to be the first!
\addpublication[conference]{\underline{Pedram Daee}, Joel Pyykk\"o, Dorota G\l{}owacka, and Samuel Kaski}{Interactive Intent Modeling from Multiple Feedback Domains}{Proceedings of the 21st International Conference on Intelligent User Interfaces}{Sonoma, California, USA, 71--75}{March}{2016}{ACM}{c1}

%\addpublication{Journal Paper Authors}{Journal Paper Title}{Journal Name}{Volume, issue, pages, and other detailed information}{Month}{Year}{Copyright Holder}{j1}
%\addpublication[conference]{Conference Paper Authors}{Conference Paper Title}{Conference Name}{Location, pages, and other detailed information}{Month}{Year}{Copyright Holder}{c1}
%\addpublication[accepted]{Journal Paper 2 Authors}{Journal Paper 2 Title}{Journal Name 2}{}{Month}{Year}{Copyright Holder}{j2}
%\addpublication[submitted]{Journal Paper 3 Authors}{Journal Paper 3 Title}{Journal Name 3}{}{Submission date}{Year}{No copyright holder at this moment}{j3}
% Add the dissertation author's contribution to that publication (the order can be interchanged with \adderrata).
\addcontribution{The author had the main responsiblity in problem formulation and modeling. The author designed and implemented the simulation experiment. Joel Pyykk\"o and the author built the system for user studies and conducted them together. The author wrote the initial draft of the manuscript, after which all co-authors joined for revisions.}
% Add the errata of the publication, remove if there are none (the order can be interchanged with \addauthorscontribution).
%\adderrata{j1 I This is wrong}
% Add the publication pdf file, the filename is the parameter (must be the last).
%TODO: uncomment the following in the final version (now commented to increase speed)
%\addpublicationpdf{Articles/IUI16.pdf}



\addpublication[journal]{\underline{Pedram Daee}$^*$, Tomi Peltola$^*$, Marta Soare$^*$, and Samuel Kaski}{Knowledge elicitation via sequential probabilistic inference for high-dimensional prediction}{Machine Learning}{106, 9-10, 1599--1620}{}{2017}{}{j1}
\addcontribution{The ideas and experiments in this article were designed jointly (the first three authors contributed equally). The author had the main responsibility in the derivation of the sequential experimental design and implementation of the experiments. Dr. Tomi Peltola derived and implemented the posterior approximation. The manuscript was written jointly.}
%TODO: uncomment the following in the final version (now commented to increase speed)
%\addpublicationpdf{Articles/ML17.pdf}



\addpublication[journal]{ Iiris Sundin$^*$, Tomi Peltola$^*$, Luana Micallef, Homayun Afrabandpey, Marta Soare, Muntasir Mamun Majumder, \underline{Pedram Daee}, Chen He, Baris Serim, Aki Havulinna, Caroline Heckman, Giulio Jacucci, Pekka Marttinen, and Samuel Kaski}{Improving genomics-based predictions for precision medicine through active elicitation of expert knowledge}{Bioinformatics}{34, 13, i395–i403}{}{2018}{}{j2}
\addcontribution{The author contributed on formulation of the sequential experimental design and implementation of a portion of the early version of the experiments. The author made comments to the manuscript in preparation.}
%TODO: uncomment the following in the final version (now commented to increase speed)
%\addpublicationpdf{Articles/Bio18.pdf}


\addpublication[conference]{\underline{Pedram Daee}$^*$, Tomi Peltola$^*$, Aki Vehtari, and Samuel Kaski}{User Modelling for Avoiding Overfitting in Interactive Knowledge Elicitation for Prediction}{Proceedings of the 23rd International Conference on Intelligent User Interfaces}{Tokyo, Japan, 305--310}{March}{2018}{ACM}{c2}
\addcontribution{The ideas and experiments in this article were designed jointly (the first two authors contributed equally). The author designed and implemented the user study. Dr. Tomi Peltola had the main responsibility of the model formulation. The first two authors wrote the initial draft of the manuscript, after which all co-authors joined for revisions.}
%TODO: uncomment the following in the final version (now commented to increase speed)
%\addpublicationpdf{Articles/IUI18.pdf}


\addpublication[journal]{Giulio Jacucci, Oswald Barral, \underline{Pedram Daee}, Markus Wenzel, Baris Serim, Tuukka Ruotsalo, Patrik Pluchino, Jonathan Freeman, Luciano Gamberini, Samuel Kaski, Benjamin Blankertz}{Integrating Neurophysiological Relevance Feedback in Intent Modeling for Information Retrieval}{Journal of the Association for Information Science and Technology}{}{}{2019}{}{j3}
\addcontribution{The author had the main responsibility in design and implementation of the interactive intent modelling and information retrieval system, and writing of the correspnding sections. All the authors contributed to paper revisions.}
%TODO: uncomment the following in the final version (now commented to increase speed)
%\addpublicationpdf{Articles/JASIST19.pdf}

%% The fourth publication (yet another journal paper, submitted for publication, note the optional parameter)
%% Note that you are allowed to use this option only when submitting the dissertation for pre-examination!
% Set the publication information, detailed information is not printed
%\addpublication[submitted]{Salvatore Andolina$^*$, \underline{Pedram Daee}$^*$, Tung Vuong$^*$, Tuukka Ruotsalo, Khalil Klouche, Mats Sj\"oberg, Samuel Kaski, and Giulio Jacucci}{Proactive Entity Recommendation in Everyday Digital Tasks}{journal}{}{}{2019}{No copyright holder at this moment}{j4}
%\addcontribution{The ideas and experiments in this article were designed jointly (the first three authors contributed equally). The author had the main responsibility in design and implementation of the interactive intent modelling and entity recommendation system, and writing of the correspnding sections. All the authors contributed to paper revisions.}
% Add the publication pdf file, the filename is the parameter.
%\addpublicationpdf{Articles/XXX.pdf}

\end{document}
